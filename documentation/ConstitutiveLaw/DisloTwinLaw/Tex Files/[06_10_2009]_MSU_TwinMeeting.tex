\documentclass{beamer}

\usepackage{amsmath,amssymb,amsfonts}
\usepackage{bm}
\usepackage{array}
%\include{Shortcuts}
\newcommand{\ie}{\textit{i.e.}}
\newcommand{\eg}{\textit{e.g.}}
\newcommand{\vect}[1]{\ensuremath{\mathbf{#1}}}
\newcommand{\tensII}[1]{\ensuremath{\mathbf{#1}}}
\newcommand{\tensIV}[1]{\ensuremath{\mathbb{#1}}}
\newcommand{\slip}[1]{\ensuremath{#1^{\alpha}}}
\newcommand{\slipslip}[1]{\ensuremath{#1^{\alpha\alpha}}}
\newcommand{\slipt}[1]{\ensuremath{#1^{\tilde\alpha}}}
\newcommand{\slipslipt}[1]{\ensuremath{#1^{\alpha\tilde\alpha}}}
\newcommand{\twin}[1]{\ensuremath{#1^{\beta}}}
\newcommand{\twint}[1]{\ensuremath{#1^{\tilde\beta}}}
\newcommand{\twintwint}[1]{\ensuremath{#1^{\beta\tilde\beta}}}
\newcommand{\sliptwin}[1]{\ensuremath{#1^{\alpha\beta}}}
\newcommand{\twinslip}[1]{\ensuremath{#1^{\beta\alpha}}}

\usetheme{mpie}
\setbeamertemplate{blocks}[rounded][shadow=true]

\title{Dislocation glide and deformation twinning as implemented in DisloTwin.f90}
\date{MSU Twin Meeting, D\"usseldorf -- October 6\textsuperscript{th}, 2009}

\begin{document}

\frame{\titlepage}

\section[Outline]{}
\frame{\tableofcontents}

\section{Microstructure Parametrization}
\frame {
 \frametitle{}
 \begin{block}{\begin{center}PART I\end{center}} \begin{center}Microstructure Parametrization\end{center} \end{block}
}
\subsection{Dislocation structure}
\frame {
 \frametitle{Dislocation structure}
 \begin{block}{Internal variables:}
  \begin{itemize}
  \item<1-> $\slip N$ edge dislocation densities $\slip\varrho_{\text{edge}}$
  \item<1-> $\slip N$ dipole densities $\slip\varrho_{\text{dipole}}$
  \end{itemize}
 \end{block}

 \begin{block}{Derived measures:}
  \begin{itemize}
  \item<1-> $\slip\tau_{\text{c}}$ threshold shear stress fordislocation glide
  \item<1-> $\slip\lambda$ mean distance between 2 obstacles seen by a dislocation
  \end{itemize}
 \end{block}
}

\subsection{Mechanical twins}
\frame {
 \frametitle{Morphology and topology of mechanical twins}
 \begin{block}{Internal variables:}
  \begin{itemize}
  \item<1-> $\twin N$ twin volume fractions $\twin f$
  \item<1-> ($\twin N$ twin mean thicknesses $\twin s$)
  \end{itemize}
 \end{block}

 \begin{block}{Derived measures:}
  \begin{itemize}
  \item<1-> $f$ total twin volume fraction
  \item<1-> $\twin l=\frac{\twin s\,(1-f)}{\twin f}$ mean distance between neighboring twins $\beta$
  \item<1-> $\twin\tau_{\text{c}}$ threshold shear stress for twinning
  \item<1-> $\twin\lambda$ mean distance between 2 obstacles seen by a growing twin
  \end{itemize}
 \end{block}
}

\subsection{Threshold stresses}
\frame {
 \frametitle{Threshold stress for glide activity}
 \begin{block}{Threshold stress $\slip\tau_{\text{c}}$:}
  \begin{equation}
  \slip\tau_{\text{c}} = k_{\text{friction}}\,G_{\text{iso}}\,\sqrt{c}\,+\,G_{\text{iso}}\,\slip b\,\sqrt{\sum_{\tilde\alpha\,=\,1}^{\slip N}\,\slipslipt\xi\,(\slipt\varrho_{\text{edge}} + \slipt\varrho_{\text{dipole}})} \nonumber
  \end{equation}
 \end{block}

 \begin{block}{with:}
  \begin{itemize}
  \item<1-> $G_{\text{iso}}$ isotropic shear modulus
  \item<1-> $c$ carbon concentration (at.\%) 
  \item<1-> $k_{\text{friction}}$ adjusting parameter for solute atome friction stress
  \item<1-> $\slip b$ Burgers vector of slip system $\alpha$
  \item<1-> $\slipslipt\xi$ interaction strength (Kubin et al. 2008)
  \end{itemize}
 \end{block}
}

\frame {
 \frametitle{Threshold stress for twinning:}
 \begin{block}{Threshold stress $\twin\tau_{\text{c}}$:}
  \begin{equation}
  \twin\tau_{\text{c}} = \frac{\gamma_{\text{sfe}}}{3\,\twin b}\,+\,\frac{G_{\text{iso}}\,\twin b}{L_0} \nonumber
  \end{equation}
 \end{block}

 \begin{block}{with:}
  \begin{itemize}
  \item<1-> $\gamma_{\text{sfe}}$ temperature-dependant stacking fault energy
  \item<1-> $\twin b$ Burgers vector of twin system $\beta$
  \item<1-> $L_0$ twin source length
  \end{itemize}
 \end{block}
}

\subsection{Mean free distances}
\frame {
 \frametitle{Dislocation mean free distance between two obstacles}
 \begin{block}{Harmonic averaging:}
  \begin{eqnarray}
  \frac{1}{\slip\lambda} & = & 
  \frac{1}{d_{\text{grain}}}
  \,+\,\frac{\sqrt{\slip\varrho_{\text{edge}}\,+\,\slip\varrho_{\text{dipole}}}}{k_\lambda}
  \,+\,\frac{1}{\sliptwin d} \nonumber \\
  & = &
  \frac{1}{d_{\text{grain}}}
  \,+\,\frac{\sqrt{\slip\varrho_{\text{edge}}\,+\,\slip\varrho_{\text{dipole}}}}{k_\lambda} \,+\,\sum_{\beta\,=\,1}^{\slip{N}}\,\sliptwin{I}\,\frac{1}{\twin l} \nonumber
  \end{eqnarray}
 \end{block}

 \begin{block}{with:}
  \begin{itemize}
  \item<1-> $d_{\text{grain}}$ grain size
  \item<1-> $\sliptwin{I}$ slip--twin interactions (0 if $\alpha$,$\beta$ coplanars or cross-slip; 1 otherwise)
  \end{itemize}
 \end{block}
}

\frame {
 \frametitle{Twin mean free distance between two obstacles}
 \begin{block}{Harmonic averaging:}
  \begin{eqnarray}
  \frac{1}{\twin\lambda} & = & \frac{1}{d_{\text{grain}}} + \dfrac{1}{\twin d} \nonumber \\
  & = & \dfrac{1}{d_{\mathrm{grain}}} + \sum_{\tilde\beta\,=\,1}^{\twin{N}}\,\twintwint{I}\,\dfrac{1}{\twint l} \nonumber
  \end{eqnarray}
 \end{block}

 \begin{block}{with:}
  \begin{itemize}
  \item<1-> $\twintwint{I}$ twin--twin interactions (0 if $\beta$,$\tilde\beta$ coplanars; 1 otherwise)
  \end{itemize}
 \end{block}
}

\section{Kinetics}
\frame {
 \frametitle{}
 \begin{block}{\begin{center}PART II\end{center}} \begin{center}Kinetics\end{center} \end{block}
}
\subsection{Thermally-activated dislocation motion}
\frame {
 \frametitle{Orowan's kinetics}
 \begin{block}{Shear rate $\slip{\dot\gamma}$:}
  \begin{equation}
  \slip{\dot\gamma} = \slip\varrho_{\text{edge}}\,\slip b\,\slip v_{\text{glide}} \nonumber
  \end{equation}
 \end{block}

 \begin{block}{Velocity $\slip v_{\text{glide}}$:}
  \begin{equation}
  \slip v_{\text{glide}} = v_0\,
  \exp{\left[-\dfrac{Q}{k_{\text{B}}\,T}\,\left(1-\left(\dfrac{|\slip\tau|}{\slip\tau_{\text{c}}}\right)^p\right)^q\right]} \operatorname{sign}(\slip\tau) \nonumber
  \end{equation}
 \end{block}

 \begin{block}{with:}
  \begin{itemize}
  \item<1-> $v_0$ velocity pre-factor
  \item<1-> $Q$ activation energy for dislocation glide
  \item<1-> $k_{\text{B}}\,T$ Boltzmann energy
  \end{itemize}
 \end{block}
}

\subsection{Twin kinetics}
\frame {
 \frametitle{Twin nucleation law}
 \begin{block}{Shear rate $\twin{\dot\gamma}$:}
  \begin{equation}
  \twin{\dot\gamma} = \twin\gamma_{\text{c}}\,\twin{\dot f} = \twin\gamma_{\text{c}}\,(1-f)\,\twin V\,\twin{\dot N} \nonumber
  \end{equation}
 \end{block}

 \begin{block}{Nucleation rate $\twin{\dot N}$:}
  \begin{equation}
  \twin{\dot N} = \dot{N}_0\,\exp\left[-\left(\frac{\twin\tau_{\text{c}}}{\twin\tau}\right)^r\right] \nonumber
  \end{equation}
 \end{block}

 \begin{block}{with:}
  \begin{itemize}
  \item<1-> $\twin\gamma_{\text{c}}$ characteristical twin shear
  \item<1-> $\twin V$ volume of grown-up twins
  \item<1-> $\dot{N}_0$ constant twin nucleation rate per time and volume
  \end{itemize}
 \end{block}
}

\frame {
 \frametitle{Spontaneous twin growth}
 \begin{block}{Volume of grown-up twins $\twin V$:}
  \begin{equation}
  \twin V = \frac{\pi}{6}\,\twin s\,{\twin\lambda}^2 \nonumber
  \end{equation}
 \end{block}
}

\section{Evolution laws for microstructure}
\frame {
 \frametitle{}
 \begin{block}{\begin{center}PART III\end{center}} \begin{center}Evolution laws for microstructure\end{center} \end{block}
}
\subsection{Multiplication and annihilation mechanisms}
\frame {
 \frametitle{Dislocation multiplication}
 \begin{block}{Multiplication:}
  \begin{equation}
  \slip{\dot\varrho_{\text{multiplication}}} = \dfrac{|\slip{\dot\gamma}|}{\slip b\,\slip\lambda} \nonumber
  \end{equation}
 \end{block}
}

\frame {
 \frametitle{Dipole formation}
 \begin{block}{Dipole formation:}
  \begin{equation}
  \slip{\dot\varrho_{\text{formation}}} = 2\,\dfrac{2\,\operatorname{max}(\slip{\hat d},\slip{\check d})}{\slip b}\,\dfrac{\slip\varrho_{\text{edge}}}{2}\,|\slip{\dot\gamma}| \nonumber
  \end{equation}
 \end{block}

 \begin{block}{Upper stability limit for dipoles $\slip{\hat d}$:}
  \begin{equation}
  \slip{\hat d} = \dfrac{1}{8\,\pi}\,\dfrac{G_{\text{iso}}\,\slip b}{1-\nu}\,\dfrac{1}{|\slip\tau|} \nonumber
  \end{equation}
 \end{block}
}

\frame {
 \frametitle{Spontaneous annihilation of 2 single dislocations}
 \begin{block}{Single--single annihilation:}
  \begin{equation}
  \slip{\dot\varrho_{\text{single--single}}} = 2\,\dfrac{2\,\slip{\check d}}{\slip b}\,\dfrac{\slip\varrho_{\text{edge}}}{2}\,|\slip{\dot\gamma}| \nonumber
  \end{equation}
 \end{block}

 \begin{block}{Lower stability limit of dipoles $\slip{\check d}$:}
  \begin{equation}
  \slip{\check d} \propto \slip b \nonumber
  \end{equation}
 \end{block}
}

\frame {
 \frametitle{Spontaneous annihilation of one single dislocation with a dipole constituent}
 \begin{block}{Single--dipole constituent annihilation:}
  \begin{equation}
  \slip{\dot\varrho_{\text{single--dipole}}} = 2\,\dfrac{2\,\slip{\check d}}{\slip b}\,\dfrac{\slip\varrho_{\text{dipole}}}{2}\,|\slip{\dot\gamma}| \nonumber
  \end{equation}
 \end{block}
}

\frame {
 \frametitle{Dipole climb}
 \begin{block}{Dipole climb:}
  \begin{equation}
  \slip{\dot\varrho_{\text{climb}}} = \slip\varrho_{\text{dipole}}\,\dfrac{2\,v_{\text{climb}}}{(\slip{\hat d}-\slip{\check d})/2} \nonumber
  \end{equation}
 \end{block}

  \begin{block}{Climb velocity $\slip v_{\text{climb}}$:}
  \begin{equation}
  \slip v_{\text{climb}} = \dfrac{D\,\slip\Omega}{\slip b\,k_{\text{B}}\,T}\,\dfrac{G_{\text{iso}}\,\slip b}{2\,\pi\,(1-\nu)}\,\dfrac{1}{(\slip{\hat d}+\slip{\check d})/2} \nonumber
  \end{equation}
 \end{block}
}

\subsection{Evolution of dislocation densities}
\frame {
 \frametitle{Evolution of dislocation densities}
 \begin{block}{Edge dislocation density rate:}
  \begin{equation}
  \slip{\dot\varrho_{\text{edge}}} = \slip{\dot\varrho_{\text{multiplication}}} - \slip{\dot\varrho_{\text{formation}}} - \slip{\dot\varrho_{\text{single--single}}} \nonumber
  \end{equation}
 \end{block}

\begin{block}{Dislocation dipole density rate:}
  \begin{equation}
  \slip{\dot\varrho_{\text{dipole}}} = \slip{\dot\varrho_{\text{formation}}} - \slip{\dot\varrho_{\text{single--dipole}}} - \slip{\dot\varrho_{\text{climb}}} \nonumber
  \end{equation}
 \end{block}
}

\end{document} 