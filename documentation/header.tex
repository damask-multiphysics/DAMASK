%\NeedsTeXFormat{LaTeX2e} 
%%% check whether we are running pdflatex
% http://www.db.informatik.uni-bremen.de/~mr/pdflatex.html
%	\newif\ifpdf
%	\ifx\pdfoutput\undefined
%	  \pdffalse % we are not running pdflatex
%	\else
%	  \pdfoutput=1 % we are running pdflatex
%	  \pdfcompresslevel=9     % compression level for text and image;
%	  \pdftrue
%	\fi

%%% choose global options depending on whether
%%% we are running under pdflatex
	
%	\ifpdf
%\ifx\pdfoutput
%\documentclass[pdftex,a4paper,10pt,makeidx]{article}
%\documentclass[pdftex,a4paper,10pt,makeidx]{book}
%\documentclass[pdftex,a4paper,12pt]{book}
%\documentclass[pdftex,a4paper,12pt]{report}
%\documentclass[pdftex,a5paper,9pt]{book}
%\documentclass[pdftex,a4paper,11pt]{book}
\documentclass[pdftex,a4paper,12pt]{book}   % DISS
%\documentclass[pdftex,a4paper,12pt]{article} % List of publications
%	\else
%\fi
%\ifx\pdfoutput\undefined
%	  \documentclass[dvips,a4paper,10pt,makeidx]{article}
%\fi

% WINDOWS: use sumatra pdfviewer with TeXnicCenter to do inverse search
%http://www.wi.uni-muenster.de/qm/studieren/LaTeXVorlage.html
\usepackage{pdfsync}  % comment out for final document

\usepackage{ifpdf}  % uses oberdiek, cannot be used before documentclass
\usepackage{etex}  % Without there will be many "No room for a new \dimen" errors
\usepackage{a4}
\usepackage[T1]{fontenc}                      % EC-Schriften("richtige"Umlaute)
\usepackage[latin1]{inputenc}                 % Input coding
%\usepackage[utf8]{inputenc}                 % Input coding

%\renewcommand\appendixname{}%    "
\renewcommand\figurename{Fig.}
\renewcommand\tablename{Tab.}
%\usepackage{csquotes} % consistent quoting by \enquote{...}, without use \lq xyz\rq or \lqq xyz \rqq
% TABLES
%\usepackage{array,booktabs}
\usepackage{upgreek}
\usepackage{ulem} % provides \sout  strikeout
\normalem % restore italics for emph   
\usepackage{pifont}  % checkmark = \ding{52}
\usepackage{textcomp} % provides \textnumero
\usepackage{txfonts} % gammaup, upright greek characters
\usepackage{booktabs}       % Nicer Tables, provides \toprule, \midrule, \bottomrule
\usepackage{longtable}
\usepackage{multirow}% added090731, \multirow{3}{*}{content}, * for width
%\usepackage{ltablex}
%\usepackage{tabulary}% 
%\usepackage{ltxtable}% longable meet tabularx
%\usepackage{tools}

%\usepackage{sfheaders}              % SansSerif Font f�r �berschriften / for headings
\usepackage{pictex}                 % Plot-Paket
\usepackage{anysize}               % Uses full page, all of papersize, looks bad!
%\usepackage{fullpage}
\usepackage{xspace}                 % provides \xspace
%\usepackage[pdftex]{lscape}                 % Provides \begin{landscape} environment
\usepackage{pdflscape}                 % Provides \begin{landscape} environment
%%%%%%%%%%%%%%%%%%%%%
\usepackage{graphicx}               % PS-diagram package of pictures merge
%\usepackage[draft]{graphicx}        % PS-Grafikpaket nur Rahmen
%%%%%%%%%%%%%%%%%%%%%
\usepackage{psfrag}                 % PS-Text durch LaTeX-Code ersetzen
%\usepackage{draftstamp}    %%%%%%%%%% Entwurfstempel; draftstamp.sty %%%%%%%%%%%%%%%%%
\usepackage{placeins}        % definiert Floatsperre \FloatBarrier; placeins.sty
\usepackage[indent,bf]{caption}    % Einstellen des caption-Stils, war caption2
%\captionindent2em                   % Ma� f�r Einzug
% Workaround for backcite problem 
\usepackage{makerobust} 
\makeatletter 
\@ifundefined{caption@xref}{}{% 
  \MakeRobustCommand\caption@xref 
} 
\makeatother 
%I have notified the author of the caption package. 
%Yours sincerely 
%  Heiko <oberd...@uni-freiburg.de>

% Workaround against double definition error by using amssymb and txfonts:iint
\let\iint\dadada
\let\iiint\dadadaa
\let\iiiint\dadadab
\let\idotsint\dadadac

\usepackage{exscale}                % richtig skalierte Operatoren in Formeln
\usepackage{amsmath}                % Mathematik-Erweiterungen der AMS
\usepackage{amssymb}                % Mathematik-Erweiterungen der AMS
%\usepackage{onlyamsmath}
% ftp://tug.ctan.org/pub/tex-archive/info/Free_Math_Font_Survey/survey.html
%\usepackage{concmath}                %
%\usepackage{mathtime}
%\usepackage{mnsymbol}                 % incompatible with amssymb & amsfonts, pixelig, neswarrows
%\usepackage{cmbright}
%\usepackage{mathpazo}								% funktioniert
\usepackage{times}                  % von elsart abgeschaut
%\usepackage[math]{anttor}
%\usepackage{millennial}
%\usepackage{fouriernc}              % sch�neres gamma
%\usepackage{yhmath}								%
\usepackage{mathrsfs}               % for mathscr (Schreibschrift)
%\usepackage[math]{isomath}

%\ptexnoligatures%\textfont0 % bewirkt angebl. Unterdr�ckung von Ligaturen und kerning in pdftex

\usepackage{afterpage}              % Befehlsausf�hrung erst am Seitenende
\usepackage{natbib}                 % ein natbib-Stil, der auch bei author-year eine
% biblabel Marke im Literaturverzeichnis aufweist!
% natbib.sty ruft eine eigene Konfiguration natbib.cfg auf

\usepackage{xcolor}
\definecolor{lbcolor}{rgb}{0.9,0.9,0.9}
\definecolor{lightgrey}{rgb}{0.85,0.85,0.85} %
\definecolor{gray}{rgb}{0.5,0.5,0.5}% gray=AE, grey=BE
\definecolor{grey}{rgb}{0.5,0.5,0.5}% gray=AE, grey=BE

\usepackage{upgreek}

% KOPF- UND FUSSZEILEN
\usepackage{fancyhdr}        % Erwieterte Optionen f�r
\pagestyle{fancy}            % Kopf- und Fu�zeilen
\fancyhf{}                   % Alle Felder leeren
%\fancyfoot{}\fancyhead{}     %CZ clear all fields
%\fancyfoot[ro,le]{\footnotesize\thepage}  % Seitenzahl au�en
%\fancyfoot[l]{\footnotesize\today}
% NUR SEITENZAHL
\fancyfoot[ro,le]{\footnotesize\thepage}%\\\textcolor{gray}{C.\,Zambaldi, \today}}   % Seitenzahl
% SEITENZAHL UND VERSION INFO
%\fancyfoot[ro,le]{\footnotesize\thepage\\\textcolor{gray}{C.\,Zambaldi, \today}}   % Seitenzahl

\fancyhead[ro]{\footnotesize{\leftmark} }    % Kapitel rechts auf geraden Seiten
\fancyhead[le]{\footnotesize{\rightmark} }   % Abschnitt links auf ungeraden Seiten%
%\fancyhead[ro]{\textsc{\small{\leftmark}}}  %Zambaldi  % Kapitel rechts auf geraden Seiten
%\fancyhead[le]{\textsc{}} %2005  % Abschnitt links auf ungeraden Seiten%
\setlength{\headheight}{15pt}% H�he der Kopfzeile (def012pt) wg der Linie
\renewcommand{\headrulewidth}{0pt} %def0.4pt
\renewcommand{\footrulewidth}{0pt} %def0pt
\renewcommand\chaptername{}      %fancyhdr: Kapitel�berschrift ohne "Kapitel"
%\renewcommand{\chaptermark}[1]{\markboth{\textsc{\chaptername\ \thechapter\ \ #1}}{}}
%\renewcommand{\chaptermark}[1]{\markboth{{\thepart-\thechapter\ \ #1}}{}}
\renewcommand{\chaptermark}[1]{\markboth{{\thechapter\ \ #1}}{}}
%\renewcommand{\chaptermark}[1]{\markboth{{\thechapter\ }}{}}
%\renewcommand{\chaptermark}{\empty}
%\markboth{Zambaldi}
\renewcommand{\sectionmark}[1]{\markright{{\thesection\ #1}}}% Kein Punkt hinter Kapitelnummern in Kopfzeile
%\renewcommand{\sectionmark}{\empty}
%\renewcommand{\chaptermark}[1]{}
%\renewcommand{\sectionmark}[1]{}
\fancypagestyle{plain}{% Redefine "plain" style (e.g. for 1st side of chapters)
\fancyhf{} % clear all header and footer fields
%\fancyfoot[l]{{\tiny \today}}
%\fancyfoot[ro,le]{\footnotesize\thepage}
}
% http://www-h.eng.cam.ac.uk/help/tpl/textprocessing/squeeze.html
%\usepackage[medium,compact]{titlesec} % fontsize sections...
\usepackage[small,compact]{titlesec} % fontsize sections...
%\usepackage{sectsty} % An alternative to titlesec

%\titleformat{\subsection}{\normalfont\normalsize\bfseries}{\thesubsection}{1em}{} %bfseries to add bold style
%\titleformat{\subsubsection}%{\normalfont\normalsize\mdseries\itshape}{\thesubsubsection}{1em}{}
%\titleformat{\subsubsection}{\normalfont\normalsize\bfseries}{\thesubsubsection}{1em}{}


% TABELLENBESCHRIFTUNG
%\usepackage[tableposition=top]{caption}

% EINHEITEN
\usepackage{units}
% \unit[<wert>]{<einheit>}
% \unitfrac[<wert>]{<z�hler>}{<nenner>} 	
% \nicefrac[<schrift>]{<z�hler>}{<nenner>}

% MILLERSCHE INDICES
\usepackage{miller} % \hkl(123), \hkl<1 2 -3>, change spacing, with 
%\renewcommand{\millerskip}{\,\,\,}

% DOPPELTER ZEILENABSTAND mit \doublespace, weitere
\usepackage{setspace}
\singlespace % \singlespace, \onehalfspace, \doublespace

% LISTINGS; Quellcode einbinden und formatieren.
\usepackage{listings}       %[savemem]{listings}
\lstloadlanguages{fortran} %TeX %[LaTeX]TeX
\lstset{language={},
  frame=none,
  xleftmargin=10mm,
  xrightmargin=10mm,
	numbers=left,
	stepnumber=1,
	numbersep=5pt,
	numberstyle=\tiny,
	breaklines=true,
	breakautoindent=true,
	postbreak=...,%\space,
	tabsize=2,
	basicstyle=\ttfamily\footnotesize,
	showspaces=false,
	showstringspaces=false,
	extendedchars=true,
	backgroundcolor=\color{white}
}

% ---------------------------------------------------------------------------

% BILD�BERLAGERUNG
%\usepackage[abs]{overpic} % [percent]

%\usepackage{calc}

% PARAGRAPH INDENTATION AND VERTICAL SPACE
\setlength{\parindent}{0pt}                              % kein Einzug am Absatz
%\parskip 1.0ex plus 0.2ex minus 0.2ex                    % Abstand Absatz FR:1.5ex plus 0.5ex minus 0.5ex
\setlength\parskip{.5\baselineskip
	plus .1\baselineskip
	minus .4\baselineskip
}

\sloppy                                                  % Seitenende "schlampig"
\setcounter{secnumdepth}{3}                              % Abschnitts-Num.tiefe(def=2)
\setcounter{tocdepth}{3}                       % Inhaltsverzeichnistiefe (def=2)
\setcounter{bottomnumber}{2}                   % Max. Gleitobjekte unten (def=1)
\setcounter{totalnumber}{4}                    % Max. Gleitobjekte insg.  (def=3)
\renewcommand{\topfraction}{0.75}              % Max. rel. Platz f�r Gleitobjekte
\renewcommand{\bottomfraction}{0.65}           % oben bzw. unten (def=0.7 bzw. 0.3)
\newcommand{\versetzung}[1]{\rotatebox{#1}{\rule[-2mm]{0cm}{4mm}\tiny $\bot$}}    % Versetzungssymbol{Winkel}
\newcommand{\vernull}{\tiny $\bot$}            % Versetzungssymbol 0 Grad
\newcommand{\vereinsachtnull}{\tiny $\top$}    % Versetzungssymbol 180 Grad
\newcommand{\verneunnullp}{\tiny $\dashv$}     % Versetzungssymbol 90 Grad
\newcommand{\verneunnullm}{\tiny $\vdash$}     % Versetzungssymbol -90 Grad
%\newcommand{\RX}{Rekristallisation}
%, als Dezimaltrennzeichen in Kommazahlen im Mathemodus
\mathchardef\CommaOrdinary="013B
\mathchardef\CommaPunct="613B
\mathcode`,="8000 %, im Math-Mode aktiv machen
{\catcode`\,=\active
 \gdef,{\obeyspaces\futurelet\next\CommaCheck}}
\def\CommaCheck{\if\space\next\CommaPunct\else\CommaOrdinary\fi}

%\def\figurename{} % Fig. oder Abb.								
\makeatletter                                                   % @ define as normal letters
\def\cleardoublepage{\clearpage \if@twoside \ifodd\c@page       % with CLEAR double PAGE second side empty
                     \else \hbox{} \thispagestyle{empty} \newpage
                     \if@twocolumn \hbox{} \newpage \fi \fi \fi}
%\renewcommand\part{%               % with part first side empty
%  \if@openright
%    \cleardoublepage
%  \else
%    \clearpage
%  \fi
%  \thispagestyle{empty}%
%  \if@twocolumn
%    \onecolumn
%    \@tempswatrue
%  \else
%    \@tempswafalse
%  \fi
%  \null\vfil
 % \secdef\@part\@spart}
%\makeatother         % @ again as other indication define

% Upright greek characters: http://www.superstrate.net/useful/useful.html
\DeclareFontFamily{U}{euc}{}% I chose euc because the chart is called Euler cursive 
\DeclareFontShape{U}{euc}{m}{n}{<-6>eurm5<6-8>eurm7<8->eurm10}{}% 
\DeclareSymbolFont{AMSc}{U}{euc}{m}{n} % I chose AMSc because AMSa and AMSb are defined in the amsfonts-package 
\DeclareMathSymbol{\umu}{\mathord}{AMSc}{"16}


% HYPERREF   package definition
\usepackage{color}
\definecolor{lnkcol}{rgb}{0, .1, .5}
\definecolor{lnkcol}{rgb}{0, .1, .5}


%\ifpdf

%\usepackage{pstricks, pst-plot}

\ifpdf   % IFPDF
  %\usepackage[pdftex]{thumbpdf}      %%% thumbnails for ps2pdf
%\usepackage[ps2pdf]{hyperref}
  \usepackage[pdftex,                %%% hyper-references for ps2pdf
%backref=section,											 %%% bibliographical backlinks
pagebackref=true,	                 %%% links in bibliography to pages
bookmarks=true,%                   %%% generate bookmarks
bookmarksnumbered=false,           %%% numbered bookmarks on/off
bookmarksopen=true,                %%% bookmarkstree open/closed
hypertexnames=false,%              %%% needed for correct links to figures!
breaklinks=true,%                  %%% break lines on/off, on => links can become very small
pdfborder=false,                   %%% Links mit Rahmen on/off
pdfpagelabels=true,								 %%% logische/physikalische Seitenzahlen true/false*
pdftoolbar=true,									 %%% Men�leisten true*/false	
linkbordercolor={1 1 1},%    %%% color of frames around links
citebordercolor={1 1 1},
urlbordercolor={0.8 0.8 .8},
colorlinks=true,
linkcolor=lnkcol, % black, blue, red, green, cyan, ...
urlcolor=lnkcol,%{.2 .2 1},
citecolor=black,
]{hyperref} %HYPERREF

% PDF METAINFORMATION
  \hypersetup{
pdfauthor   = {},
pdftitle    = {},
pdfsubject  = {},
pdfkeywords = {},
pdfcreator  = {},
pdfproducer = {pdftex}%,
%pdfpagemode={FullScreen}
}

% ATTACH ARBITRAY FILES TO THE RESULTING PDF. FILES ARE STORED INSIDE OF THE PDF-FILE.
%\usepackage{attachfile}  % Only with pdftex. Makes use of "oberdiek" package
%\attachfilesetup{color=0 .1 .5}


  \usepackage{pst-pdf} % pstricks for pdflatex

\else
  \usepackage{pstricks, pst-plot}
  \usepackage[hypertex]{hyperref}
\fi

\usepackage{textcomp} %Dieses Paket enthaelt die Befehle %\textdegree%\textcentigrade%und eine Menge anderer nuetzlicher Symbole.


% MULTIPLE BIBLIOGRAPHIES
%\usepackage{bibunits}

%\usepackage{makeidx}
%\makeindex

%%%%%%%%%%%%%%%%%%%%%%%%%%%%%%%%%%%%%%%%%%%%%%%%%%%%%%%%%
%%%%%%%%%%% END OF HEADERFILE %%%%%%%%%%%%%%%%%%%%%%%%%%%
%%%%%%%%%%%%%%%%%%%%%%%%%%%%%%%%%%%%%%%%%%%%%%%%%%%%%%%%% 